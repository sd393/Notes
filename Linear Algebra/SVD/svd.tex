\documentclass{article}
\usepackage{fancyhdr}
\usepackage{amsmath}
\usepackage{parskip}
\usepackage{tikz}
\usepackage{tcolorbox}
\pagestyle{fancy}

\begin{document}
Singular value decomposition (SVD) generalizes matrix diagonalization to non-square matrices. 
It is powerful because it applies to \textbf{any} $m \times n$ matrix.

First we give the recipe: 
\begin{equation} 
    A = U \Sigma V^T. \label{eq:SVD} \tag{SVD}
\end{equation}
Compute $AA^T$. This product is guaranteed to have $m$ linearly independent eigenvectors by the 
\textbf{spectral theorem}. Collate those eigenvectors into the columns of a matrix. That matrix is $U$. 

Then, compute $A^TA$, which is guaranteed to have $n$ linearly independent eigenvectors 
that you can use to create $V$.

Next compute the non-zero eigenvalues $\lambda_i$ of $A^TA$, take their square roots 
$\sigma_i = \sqrt{\lambda_i}$, and stick them into the diagonal of $\Sigma$. 

Now $\Sigma$ will consist of $\text{rank}(A)$ $\sigma_i$'s on the diagonal and zeroes elsewhere.
Add more zeroes on the diagonal to form a $\min(m, n) \times \min(m, n)$ square matrix. Then add the sufficient
amount of rows/columns to form a $m \times n$ matrix---obviously the rows/columns consist 
entirely of zeroes.

We are almost done. Conventionally, the singular values in $\Sigma$ are ordered along the diagonal
from greatest to least. This orderering of singular values should correspond to the ordering of 
eigenvectors in $U$ and $V$ (i.e. $Av_i = \sigma_iu_i$).

Next we look at the intuition. The first perspective we'll look at is SVD as a way of encoding linear transformations.
For example, imagine you had any square defined by the two vectors at its sides---say $\vec{v_1}$
and $\vec{v_2}$. Left-multiplying these vectors by some $A$ would then apply some linear transformation
to the unit square. Here's the crucial claim of SVD: \textbf{any such linear transformation is equivalent
to a rotation and a scaling, provided you are able to rotate the vector first.} More specifically the 
claim is 
\begin{align}
MR_1 &= R_2\Sigma \Rightarrow \nonumber \\
M &= R_2 \Sigma R_1^T \nonumber
\end{align}
which comes from right multiplication by $R_1^T$ (which is orthogonal). This seems obvious for many
transformations, such as rotations, scalings, and translations, but surprisingly it
is true for shearing as well. Even more surprisingly, this holds true for transformations that change
the dimension of the input vectors.

I like the other perspective as well. The SVD of $A$ represents the action of $A$ in the basis consisting
of the eigenvectors of $AA^T$. It's probably useful to take a step back and look at the analog for 
matrix diagonalization.

\begin{tcolorbox}[colback=gray!5!white,colframe=gray!75!black,title=Matrix Diagonalization Analog]
  Consider the standard basis and the basis defined by the eigenvectors of a diagonalizable $A$, which
  we'll call $B_1$ and $B_2$ respectively. In $B_2$, the matrix $\Lambda$ transforms the vector coordinates
  of the point $P$ to $P'$. When $P$ is expressed using the standard basis vectors in $B_1$, what matrix $A$ transforms
  $P$ to $P'$?
  \medskip
  \newline
  To answer this question first take the vector coordinates representing $P$ in the standard basis and plop
  them onto $B_2$ to get some (probably totally) different point $Q$. It turns out that $S^{-1}$, where
  the columns of $S$ consist of the eigenvectors of $A$, transforms $Q$ to $P$. And as stated before
  $\Lambda$ transforms $P$ to $P'$.
  \medskip
  \newline
  We're not done though. We know how to take a set of coordinates representing $P$ in $B_1$ and turn 
  them into a set of coordinates representing $P'$ in $B_2$. We want to represent $P'$ in $B_1$. For
  this reason we left-multiply by $S$---this results in $P'$ going to some other different point $W$ 
  in $B_2$, but it turns out that if you take the vectors representing this $W$ in $B_2$ and plop
  them onto $B_1$ you get $P'$. 
  \medskip
  \newline
  All of this is to say that $A = S \Lambda S^{-1}$ is the matrix that transforms $P$ to $P'$ in $B_1$,
  given that $\Lambda$ transforms $P$ to $P'$ in $B_2$.
\end{tcolorbox}

Now that we understand the matrix diagonalization analog, we can view SVD as a transformation in a new 
basis of vectors.

\end{document}