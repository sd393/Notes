\documentclass{article}
\usepackage{fancyhdr}
\usepackage{amsmath}
\usepackage{parskip}
\pagestyle{fancy}

\begin{document}
Singular value decomposition (SVD) generalizes matrix diagonalization to non-square matrices. 
It is powerful because it applies to \textbf{any} $m \times n$ matrix.

First we give the recipe: 
\begin{equation} 
    A = U \Sigma V^T. \label{eq:SVD} \tag{SVD}
\end{equation}
Compute $AA^T$. This product is guaranteed to have $m$ linearly independent eigenvectors by the 
spectral theorem. Collate those eigenvectors into the columns of a matrix. That matrix is $U$. 

Similarly, compute $A^TA$, which is guaranteed to have $n$ linearly independent eigenvectors 
that you can stick into $V$.

Next compute the non-zero eigenvalues $\lambda_i$ of $A^TA$, take their square roots 
$\sigma_i = \sqrt{\lambda_i}$, and stick them into the diagonal of $\Sigma$. 

Now $\Sigma$ will consist of $\text{rank}(A)$ $\sigma_i$'s on the diagonal and zeroes elsewhere.
Add more zeroes on the diagonal to form a $\min(m, n) \times \min(m, n)$ square matrix. Then add the sufficient
amount of rows/columns to form a $m \times n$ matrix - obviously the rows/columns consist 
entirely of zeroes.


\end{document}